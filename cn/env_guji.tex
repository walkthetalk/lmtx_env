%
% author:	Ni Qingliang
% date:		2011-02-11
%
\startenvironment env_guji

\input e_base
\input e_lang
\input e_color
\input e_lang_cht
\input e_font
\input e_font_guji
\input e_layout_guji

%%%%%%%%%%%%% 使用模块(保持顺序) %%%%%%%%%%%%%
% 竖排
\usemodule[t][vtypeset]


% 标点压缩与支持
\usemodule[t][zhpunc][pattern=kaiming, spacequad=0.5, hangjian=false]
% 
% 四种标点压缩方案:全角、开明、半角、原样:
%   pattern: quanjiao(default), kaiming, banjiao, yuanyang
% 行间标点(转换`、,。.:!;?`到行间,pattern建议用banjiao):
%   hangjian: false(default), true
% 加空宽度(角):
%   spacequad: 0.5(default)
% 
% 行间书名号和专名号(\bar实例):
%   \zhuanmh{专名}
%   \shumh{书名}


% 夹注
\usemodule[t][jiazhu][fontname=tf, fontsize=10.5pt, interlinespace=0.2em]
% default: fontname=tf, fontsize=10.5pt, interlinespace=0.08em(行间标点时约0.2em)
% fontname和fontsize与\switchtobodyfont的对应参数一致
% 夹注命令:
%   \jiazh{夹注}


%%%%% 显式视觉调试信息 %%%%
% \showboxes
% \showglyphs
% \showframe
% \showmakeup
% \tracingnodes1 %1,2

\stopenvironment
