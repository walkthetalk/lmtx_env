
% 字号对应表
\definemeasure[fs0][42pt]	% 初号
\definemeasure[fs0s][36pt]	% 小初
\definemeasure[fs1][26pt]	% 一号
\definemeasure[fs1s][24pt]	% 小一
\definemeasure[fs2][22pt]	% 二号
\definemeasure[fs2s][18pt]	% 小二
\definemeasure[fs3][16pt]	% 三号
\definemeasure[fs3s][15pt]	% 小三
\definemeasure[fs4][14pt]	% 四号
\definemeasure[fs4s][12pt]	% 小四
\definemeasure[fs5][10.5pt]	% 五号
\definemeasure[fs5s][9pt]	% 小五
\definemeasure[fs6][7.5pt]	% 六号
\definemeasure[fs6s][6.5pt]	% 小六
\definemeasure[fs7][5.5pt]	% 七号
\definemeasure[fs8][5pt]	% 八号

\definebodyfontenvironment [9pt] [
	xx=6.5pt,x=7.5pt,text=9pt,a=10.5pt,b=12pt,c=14pt,d=15pt,
	scriptscript=5.5pt,script=7.5pt,small=7.5pt,big=10.5pt
]
\definebodyfontenvironment [10.5pt] [
	xx=7.5pt,x=9pt,text=10.5pt,a=12pt,b=14pt,c=15pt,d=16pt,
	scriptscript=7.5pt,script=9pt,small=9pt,big=12pt
]
\definebodyfontenvironment [12pt] [
	xx=9pt,x=10.5pt,text=12pt,a=14pt,b=15pt,c=16pt,d=18pt,
	scriptscript=9pt,script=10.5pt,small=10.5pt,big=14pt
]
\definebodyfontenvironment [14pt] [
	xx=10.5pt,x=12pt,text=14pt,a=15pt,b=16pt,c=18pt,d=22pt,
	scriptscript=9pt,script=12pt,small=12pt,big=15pt
]
\definebodyfontenvironment [15pt] [
	xx=12pt,x=14pt,text=15pt,a=16pt,b=18pt,c=22pt,d=24pt,
	scriptscript=9pt,script=12pt,small=12pt,big=18pt
]
\definebodyfontenvironment [16pt] [
	xx=14pt,x=15pt,text=16pt,a=18pt,b=22pt,c=24pt,d=26pt,
	scriptscript=10.5pt,script=14pt,small=14pt,big=18pt
]
\definebodyfontenvironment [18pt] [
	xx=15pt,x=16pt,text=18pt,a=22pt,b=24pt,c=26pt,d=36pt,
	scriptscript=10.5pt,script=12pt,small=14pt,big=22pt
]
\definebodyfontenvironment [22pt] [
	xx=16pt,x=18pt,text=22pt,a=24pt,b=26pt,c=36pt,d=42pt,
	scriptscript=12pt,script=16pt,small=18pt,big=24pt
]

% zhfonts
\usemodule[t][zhfonts][style=rm, size=\measure{fs4}]
%\setupzhfonts[feature][onum=yes, pnum=yes]

% note: 中文部分以noto-cjk爲基礎,斜體用Adobe的中文字體,粗斜體全部用Adobe的黑體
\setupzhfonts
[serif]
[regular=notoserifcjksc,%adobesongstd,
bold=notoserifcjkscbold,%adobeheitistd,
italic=adobesongstd,
bolditalic=adobeheitistd]

\setupzhfonts
[sans]
[regular=notosanscjksc,%adobefangsongstd,
bold=notosanscjkscblack,%adobeheitistd,
italic=adobefangsongstd,
bolditalic=adobeheitistd]

\setupzhfonts
[mono]
[regular=notosansmonocjksc,%adobekaitistd,
bold=notosansmonocjkscbold,%adobeheitistd,
italic=adobekaitistd,
bolditalic=adobeheitistd]

\setupzhfonts
[latin, serif]
[regular=notoserifregular,%texgyrepagellaregular,
bold=notoserifbold,%texgyrepagellabold,
italic=notoserifitalic,%texgyrepagellaitalic,
bolditalic=notoserifbolditalic%texgyrepagellabolditalic
]

% todo: noto的mono沒有斜體,暫用adobe的source-code代替
\setupzhfonts
[latin, mono][
  regular=notosansmonoregular,%texgyrecursorregular,
  bold=notosansmonobold,%texgyrecursorbold,
  italic=sourcecodeproitalic,
  bolditalic=sourcecodeprobolditalic
]

\setupzhfonts
[latin, sans]
[regular=notosansregular,%texgyreherosregular,
bold=notosansbold,%texgyreherosbold,
italic=notosansitalic,%texgyreherositalic,
bolditalic=notosansbolditalic%texgyreherosbolditalic
]

% math
%\usetypescriptfile[euler]
%\setupzhfonts[math][pagellaovereuler]

\zhfonts[rm,\measure{fs4}]
% %D 预先设置文档主字体类型与尺寸,防止 ConTexT 将 12pt 作为 1em 的绝对长度
\setupbodyfont[zhfonts, rm, \measure{fs4}]

%%%%%%%%%%%%%%%%%%%%%%%%%%%%%%%%%%%%%%%%%%%%%%%%%% zhpunc %%%%%%%%%%%%%%%%%%%%%%%%%%%%%%%%%%%%%%%%%%%%%%%%%
%\usemodule[zhpunc][pattern=kaiming, spacequad=0.5, hangjian=false]

\definebodyfontenvironment[default][em=bolditalic]

\definealternativestyle[booktitle][{\switchtobodyfont[\measure{fs0}]\ss}][]
\definealternativestyle[booksubtitle][{\switchtobodyfont[\measure{fs1}]\ss}][]

\definealternativestyle[tblfontSmall][\ssx][]
\definealternativestyle[tblfont][\ss][]
\definealternativestyle[tblfontBig][\ssa][]
\definealternativestyle[tblfontVerybig][\ssb][]

\definealternativestyle[floatlabeltext][\rmx][]
\definealternativestyle[floatlabelhead][\bf][]

\definealternativestyle[headertext][\ss\tfx][]
\definealternativestyle[footertext][\ss\tfx][]

\definealternativestyle[titletext][\rm\bfb][]
\definealternativestyle[parttext][\rm\bfd][]
\definealternativestyle[chaptertext][\rm\bfc][]
\definealternativestyle[sectiontext][\rm\bfb][]
\definealternativestyle[subsectiontext][\rm\bfa][]
\definealternativestyle[subsubsectiontext][\rm\ita][]

\definealternativestyle[labelinmargin][\rm\tfb][]

\setuptolerance[horizontal,
  verystrict, %stretch space verystrict strict tolerant verytolerant
]
\setuptolerance[vertical,
  strict, %stretch space verystrict strict tolerant verytolerant
]
