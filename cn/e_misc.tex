\setupexternalfigures[%
  %order={pdf,png,jpg},
  location={local,global},
  %directory={../output,./output}
]

\setupmargindata[%
  %location %flushleft flushright both
  style={\rm\itx},	%normal bold slanted boldslanted type cap small... command
  %before	%command
  %after	%command
  align=left,	%inner outer left right middle normal no yes
  %line	number
  %distance	dimension
  %separator	text
  %width	dimension
  %distance	dimension
  %stack	yes no
  %see \setupframed
]

\setupfootnotes[
  %conversion=set 2,	%numbers characters Characters romannumerals Romannumerals
  %way=bytext,	%bytext bysection
  %location=high,	%page text columns firstcolumn lastcolumn high none
  %background=color,
  %backgroundcolor=red,
  %rule	on off
  %before=,	%command
  %after	command
  %width	dimension
  %height	dimension
  %bodyfont	5pt ... 12pt small big
  %style=bold,	%normal bold slanted boldslanted type cap small... command
  distance=2ex,	%dimension
  %columndistance	dimension
  %margindistance	dimension
  %n=3,		%number
  %numbercommand=\high,	%oneargument 脚注编号的位置
  %textcommand=,	%oneargument
  %split	tolerant strict verystrict number
  %textstyle	normal bold slanted boldslanted type cap small... command
  %textcolor=red,	%name
  %interaction=yes,	%yes no
  %factor	number
]

% chinese date
\startluacode
local function tochineseYear(n)
	local cap = {
		["0"] = "零",
		["1"] = "壹",
		["2"] = "贰",
		["3"] = "叁",
		["4"] = "肆",
		["5"] = "伍",
		["6"] = "陆",
		["7"] = "柒",
		["8"] = "捌",
		["9"] = "玖",
	}

	s, p = string.gsub(string.format("%d",n), "(.)", function(s) return cap[s] end)
	return s
end

function commands.chinese_date() -- wrong namespace
	local temp = os.date("*t")
	local day = temp.day

	tex.sprint(tex.ctxcatcodes,tochineseYear(temp.year))
	tex.sprint(tex.ctxcatcodes,"年")
	tex.sprint(tex.ctxcatcodes, converters.chinesecapnumerals(temp.month))
	tex.sprint(tex.ctxcatcodes,"月")

	if day % 10 == 0 then
		converters.chinesecapnumerals(day)
	else
		if day > 30 then
			tex.sprint(tex.ctxcatcodes,"卅")
		elseif day > 20 then
			tex.sprint(tex.ctxcatcodes,"廿")
		elseif day > 10 then
			tex.sprint(tex.ctxcatcodes,"拾")
		end
		day = day % 10
		if day > 0 then
			tex.sprint(tex.ctxcatcodes, converters.chinesecapnumerals(day))
		end
	end
	tex.sprint(tex.ctxcatcodes,"日")
end
\stopluacode

\define\COMPILEDATE{%
\ctxlua{commands.chinese_date()}%
}
%\setuplanguage[cn][date={year,年,month,月,day,日}]
