%%%%%%%%%%%%%%%%%%%%%%%%%%%%%%%%%%%%%%%%%% body part %%%%%%%%%%%%%%%%%%%%%%%%%%%%%%%%%%%%%%%%
\startsectionblockenvironment[bodypart]
% \note workaround for get total chapter number
\determinelistcharacteristics
   [chapter]
   [criterium=all]
%\utilitylistlength

\definelayer[specificlayer][preset=middle]
\startsetups rightlayer
\doiftextelse{\headnumber[chapter]}{
  \determineheadnumber[chapter]% @note: 对于\currentheadnumber, 不能用\headnumber[]进行计算
  \doifnot{\utilitylistlength}{0}{% @note: 運行過程中可能是 0
    \setlayerframed[specificlayer][
      x={\paperwidth-\layerwidth},
      y={\paperheight/\utilitylistlength*(2*\currentheadnumber-1)/2}, % \note: 表達式不支持小數
      frame=off,
      background=color,
      backgroundcolor=black,
      foregroundcolor=white,
      backgroundoffset=.2em,
      location={lohi,middle},
      offset=none]{%
      \labelinmargin\headnumber[chapter]%
    }
  }
}{}
\stopsetups
\startsetups leftlayer
\doiftextelse{\headnumber[chapter]}{
  \determineheadnumber[chapter]% @note: 对于\currentheadnumber, 不能用\headnumber[]进行计算
  \doifnot{\utilitylistlength}{0}{% \note: 運行過程中可能是 0
    \setlayerframed[specificlayer][
      x=0pt,
      y={\paperheight-\paperheight/\utilitylistlength*(2*\currentheadnumber-1)/2}, % \note: 表達式不支持小數
      frame=off,
      background=color,
      backgroundcolor=black,
      foregroundcolor=white,
      backgroundoffset=.2em,
      location={lohi,middle},
      offset=none]{%
      \labelinmargin\headnumber[chapter]%
    }
  }
}{}
\stopsetups
\setupbackgrounds[rightpage][setups=rightlayer,background=specificlayer]
\setupbackgrounds[leftpage][setups=leftlayer,background=specificlayer]

\resetpagenumber
%\setupheader[state=none]
\define\BodyTextLeftHeader{%
% @note location=low 保证坐在第一行上
\framedtext[frame=off,bottomframe=on,width=broad,offset=none,frameoffset=0pt,location=low]{%
    \bTABLE
    \setupTABLE[frame=off,style=headertext,offset=0em]
    %\setupTABLE[row][1][offset=0]
    \setupTABLE[column][1][align=flushright]
    \setupTABLE[column][2][align=flushleft,roffset=0.25em]
    \setupTABLE[column][3][align=flushleft]
    \setupTABLE[column][4][align={lohi,middle},width=broad]
    \bTR \bTD 文档编号\eTD\bTD :\eTD\bTD \getvariable{documentproperty}{docnumber}\eTD
      \bTD[nr=2]
        \doiftextelse{\getmarking[section][current]}
        {第\;\getmarking[chapternumber][first]\;章\;\;\getmarking[chapter][first]}
        {}
      \eTD \eTR
    \bTR \bTD 文档版本\eTD\bTD :\eTD\bTD \getvariable{documentproperty}{docversion}\eTD \eTR
    \eTABLE
}%
}
% \note 不要使用 \getnumber,使用 \getmarking
% \todo 不知如何判斷是處在 section 中,還是處在 subject 中
\define\BodyTextRightHeader{%
% @note location=low 保证坐在第一行上
\framedtext[frame=off,bottomframe=on,width=broad,offset=none,frameoffset=0pt,location=low]{%
    \bTABLE
    \setupTABLE[frame=off,style=headertext,offset=0em]
    \setupTABLE[column][2][align=flushright]
    \setupTABLE[column][3][align=flushleft,roffset=0.25em]
    \setupTABLE[column][4][align=flushleft]
    \setupTABLE[column][1][align={lohi,middle},width=broad]
    \bTR
      \bTD[nr=2]
        \doiftextelse{\getmarking[section][current]}
        {节\;\getmarking[sectionnumber][first]\;\;\getmarking[section][first]}
        {}
      \eTD
      \bTD 文档编号\eTD\bTD :\eTD\bTD \getvariable{documentproperty}{docnumber}\eTD
    \eTR
    \bTR
      \bTD 文档版本\eTD\bTD :\eTD\bTD \getvariable{documentproperty}{docversion}\eTD
    \eTR
    \eTABLE
}%
}
% @note 保证header底部对齐
\setupheader[text][before={\vskip\dimexpr(\headerheight-\lineheight)}]

\setupheadertexts[text]
[]
[\BodyTextRightHeader]
[]
[\BodyTextLeftHeader]

\define\BodyTextLeftFooter{%
\framedtext[frame=off,topframe=on,width=broad,
  offset=none,frameoffset=0pt,
  location=high,
  style=footertext]{%
\rlap{第\;\pagenumber\;页} \hfill {共\;\lastpagenumber\;页} \hfill \llap{}%
}}%
\define\BodyTextRightFooter{%
\framedtext[frame=off,topframe=on,width=broad,
  offset=none,frameoffset=0pt,
  location=high,
  style=footertext]{%
\rlap{} \hfill {共\;\lastpagenumber\;页} \hfill \llap{第\;\pagenumber\;页}%
}}%

\setupfootertexts[text]
[]
[\BodyTextRightFooter]
[]
[\BodyTextLeftFooter]

\define\BodyEdgeLeftFooter{%
\framed[frame=off]{% @note avoid wrong footer height for rotation
  \rotate[%
    rotation=-90,
    width=fit,
    height=fit,
    frame=off,
    offset=2pt,
    background=color,
    backgroundcolor=darkgray,
    foregroundcolor=white,
    corner=round,
  ]{\getvariable{documentproperty}{title} —— \from[authoremail]}
}
}
\define\BodyEdgeRightFooter{%
\framed[frame=off]{% @note avoid wrong footer height for rotation
  \rotate[%
    rotation=90,
    width=fit,
    height=fit,
    frame=off,
    offset=2pt,
    background=color,
    backgroundcolor=darkgray,
    foregroundcolor=white,
    corner=round,
  ]{\getvariable{documentproperty}{title} —— \from[authorblog]}
}
}

\setupfootertexts[edge]
[]
[\BodyEdgeLeftFooter]
[\BodyEdgeRightFooter]
[]
\stopsectionblockenvironment

%%%%%%%%%%%%%%%%%%%%%%%%%%%%%%%%%%%%%%%%%% front part %%%%%%%%%%%%%%%%%%%%%%%%%%%%%%%%%%%%%%%
\startsectionblockenvironment[frontpart]
\resetpagenumber

\setupfootertexts[text]
[]
[pagenumber]
[pagenumber]
[]

\stopsectionblockenvironment

%%%%%%%%%%%%%%%%%%%%%%%%%%%%%%%%%%%%%% appendix part %%%%%%%%%%%%%%%%%%%%%%%%%%%%%%%%%%%%%%%
\startsectionblockenvironment[appendix]
\setupfootertexts[text]
[]
[pagenumber]
[pagenumber]
[]

% \todo just workaround, front/body/appendix/back, the part number will reset internally
\setuphead[part][%
  number=no,
]
\stopsectionblockenvironment

%%%%%%%%%%%%%%%%%%%%%%%%%%%%%%%%%%%%%%% back part %%%%%%%%%%%%%%%%%%%%%%%%%%%%%%%%%%%%%%%%%
\startsectionblockenvironment[backpart]
\resetpagenumber
\stopsectionblockenvironment
